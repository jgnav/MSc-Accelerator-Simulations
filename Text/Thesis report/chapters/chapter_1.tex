\chapter{Introduction}
\label{cha:Introduction}

A Synchrotron light source is a type of x-ray source based on an electron storage ring.

At DESY in Hamburg, the PETRA IV project is upgrading the third-generation light source PETRA III to a fourth-generation light source  with ultra-low emittance in order to reach the diffraction limit. Synchrotron light sources  are used for scientific and technical purposes. The applicability of this technology extends  across a wide range of scientific disciplines, such as physics, chemistry, biology, materials  science, medicine, environmental science, and nanotechnology \cite{Schroer:426140}. 

Synchrotron light storage rings are designed to achieve very small electron beam sizes. This 
allows synchrotron radiation to be as bright as possible. It is necessary to maintain tight 
tolerances in electron beam stability to make use of these small beam sizes. To achieve 
such stability, closed orbit control is important in the design and operation of light sources. 
Electron orbit fluctuations increase electron beam size and degrade photon beam brightness, while slower orbit deviations require frequent realignment of experiments at the end of photon beam lines \cite{Safranek:1999fr}. 

Beam stability is one of the most important requirements in synchrotron light sources. Ambient vibrations and electrical noise cause orbit distortions which have to be counteracted 
by an orbit feedback \cite{sahoo2004closed}. In the case of PETRA IV, it will have a fast orbital feedback system (FOFB), with more than 700 beam position monitors and about 200 fast correctors \cite{Schroer:426140}.

The general objective of this work is to analyze the perturbations of the storage ring and simulate their effect on the feedback system.
For this purpose, it will be first necessary to implement a model of the particle motion through the lattice with a  feedback system that keeps the trajectory within tolerable margins. Then, the model of the expected perturbations with the data obtained from seismic measurements at the ring will be included. And finally the performance of the feedback system will be evaluated.

\section{Objectives}

\section{Thesis Structure}
Below, the contents of each chapter are summarised:

\begin{itemize}
    \item In Chapter 2, 
    \item In Chapter 3,
    \item In Chapter 4,
    \item In Chapter 5, 
    \item In Chapter 6,
\end{itemize}