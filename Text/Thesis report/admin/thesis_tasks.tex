 % Project description
\def\projectdescription{
	A Synchrotron light source is a type of x-ray source based on an electron storage ring.

	At DESY in Hamburg, the PETRA IV project is upgrading the third-generation light source PETRA III to a fourth-generation light source with ultra-low emittance in order to reach the diffraction limit.
	Synchrotron light sources are used for scientific and technical purposes.
	The applicability of this technology extends across a wide range of scientific disciplines, such as physics, chemistry, biology, materials science, medicine, environmental science, and nanotechnology [1].

	Synchrotron light storage rings are designed to achieve very small electron beam sizes.
	This allows synchrotron radiation to be as bright as possible.
	It is necessary to maintain tight tolerances in electron beam stability to make use of these small beam sizes.
	To achieve such stability, closed orbit control is important in the design and operation of light sources.
	Electron orbit fluctuations increase electron beam size and degrade photon beam brightness, while slower orbit deviations require frequent realignment of experiments at the end of photon beam lines [2].

	Beam stability is one of the most important requirements in synchrotron light sources.
	Ambient vibrations and electrical noise cause orbit distortions which have to be counteracted by an orbit feedback [3].
	In the case of PETRA IV, it will have a fast orbital feedback system (FOFB), with more than 700 beam position monitors and about 200 fast correctors [1].

	The general objective of this work is to analyze the perturbations of the storage ring and simulate their effect on the feedback system.
	For this purpose, it will be first necessary to implement a model of the particle motion through the lattice with feedback system that keeps the trajectory within tolerable margins.
	Then, a model of the expected perturbations with the data obtained from seismic measurements at the ring will be include,
	And finally the performance of the feedback system will be evaluated.
	}

 % Project tasks
\def\tasks{
\begin{packedenumerate}
	\item Literature review
	\item Create a software infrastructure in Python, with tools for the implementation of subsequent simulations of particle accelerators.
	For this purpose, new functionalities will be implemented with the pyAT library.
	The necessary documentation will be prepared to make the infrastructure available to other users.
	\item Identify the expected sources of disturbance.
	This will include experimental work on seismic measurements, data evaluation as well as analysis of pre-existing seismic data.
	\item The data obtained in the previous step, will be input into the simulation model of the feedback system and the effect of perturbations analyzed.
\end{packedenumerate}
}